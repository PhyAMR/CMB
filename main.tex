\documentclass{article}
\usepackage{graphicx} % Required for inserting images
\usepackage{witharrows}
\usepackage{hyperref}
\usepackage{amsmath}
\usepackage{csquotes}% Recommended
\usepackage[minnames=1,maxnames=2,maxbibnames = 3, style=authoryear, bibstyle = authoryear, backend=bibtex, giveninits=true, block = none, isbn = false, url = false, doi = false, eprint = false]{biblatex}
\usepackage{float}
\usepackage{amssymb,amsthm}
\usepackage{enumitem}

\addbibresource{refs.bib} %Imports bibliography file

\theoremstyle{definition}
\newtheorem{definition}{Definition}[section]
\newtheorem{algorithm}{Algorithm}[section]

\title{A Physicist's Guide to Analyzing CMB Correlations}
\author{Álvaro Méndez R.T.}
\date{June 2025}

\begin{document}

\maketitle
\tableofcontents

\begin{abstract}
In this article, we explore the mathematical and physical framework for analyzing the Cosmic Microwave Background (CMB) angular correlation function. We demonstrate how this function and its derived statistics can be used to test the standard $\Lambda$CDM cosmological model, potentially revealing new tensions, particularly concerning the assumption of a flat geometry.
\end{abstract}

\section{Introduction}
This work focuses on studying, from a different perspective, some tensions that have been described in recent years in relation to the Planck data and the cosmological model \textit{Flat Lambda Cold Dark Matter (F-$\Lambda$CDM)} (See \autocite{ABDALLA202249} for a review on the problems of the CMB). Specifically, this work consists of studying these discrepancies from the point of view of the angular correlation function.

The importance of this work lies in studying the discrepancies presented by the standard cosmological model using the angular correlation function, an object that has not been extensively studied in the literature. Where it has been studied, the focus has often been on correcting the function to better fit the data. Here, our objective is to use it as a tool to probe for new tensions with the standard model.

\begin{definition}[\textbf{Power Spectrum}]
  The power spectrum, $C_l$, is a fundamental tool in cosmology that describes the variance of temperature fluctuations on the celestial sphere as a function of angular scale, represented by the multipole moment $\ell$. In simple terms, it tells us the "strength" of fluctuations at different sizes. The commonly used $D_\ell$ is a dimensionless version, defined as $D_\ell = \frac{\ell(\ell+1)}{2\pi}C_\ell$.
\end{definition}

\begin{definition}[\textbf{Angular Correlation Function}]
  The angular correlation function, $C(\theta)$, measures the average product of temperature fluctuations at two points in the sky separated by an angle $\theta$. It is the Legendre transform of the angular power spectrum and provides a real-space representation of the statistical properties of the CMB.
\end{definition}

\section{Theoretical Framework and Statistics}

\subsection{From Power Spectrum to Correlation Function}
The two-point angular correlation function $C(\theta)$ is mathematically linked to the angular power spectrum $D_\ell$ through a Legendre polynomial expansion. This relationship forms the cornerstone of our analysis, allowing us to move from the harmonic space of the power spectrum to the real space of angular separations.

The function is defined as:
\begin{DispWithArrows}[format=c, displaystyle]
    C(\theta)=\sum_l\frac{2l+1}{2l(l+1)}D_\ell P_\ell(\cos(\theta))
\end{DispWithArrows}
where $P_\ell$ are the Legendre polynomials. This formula sums the contributions of all angular scales (multipoles $\ell$) to the correlation at a specific angle $\theta$.

\subsection{Statistical Estimators}
To quantify potential discrepancies between theoretical models and observational data, we employ several statistical estimators derived from the correlation function.

\subsubsection{The $S_a^b$ Statistic}
This statistic measures the integrated power of the squared correlation function over a specific angular range, from $\theta_1$ to $\theta_2$, where $a = \cos(\theta_1)$ and $b = \cos(\theta_2)$. It is particularly sensitive to the amplitude of correlations in that range.
\begin{DispWithArrows}[format=ll, displaystyle]
    S_a^b & =\int_a^b C(\theta)^2\sin(\theta)d\theta =\int_a^b \left[\sum_l \frac{2l+1}{2l(l+1)}D_\ell P_\ell(\cos(\theta))\right]^2 d\cos(\theta) \\
    & = \sum_n\sum_m \frac{2n+1}{2n(n+1)}D_n \frac{2m+1}{2m(m+1)}D_m\underbrace{\int_a^b P_n(x)P_m(x) dx}_{T_{nm}}
\end{DispWithArrows}
The calculation of the $T_{nm}$ matrix, which contains the integral of the product of two Legendre polynomials, is a key challenge. An analytical solution can be found using properties of Legendre polynomials \autocite{Carlitz1961}, which greatly speeds up the computation.
\begin{DispWithArrows}[format=ll, displaystyle]
    T_{mn} = \sum_{r=0}^{\min(m,n)} \frac{A_r A_{m-r} A_{n-r}}{A_{m+n-r}(2m+2n-2r+1)} \left[ \Delta P_{m+n-2r+1} - \Delta P_{m+n-2r-1} \right]_a^b
\end{DispWithArrows}
where $\Delta P_k = P_k(b) - P_k(a)$ and $A_{r}=\frac{(2r-1)!!}{r!}$.

\subsubsection{The $\bar{\langle\xi\rangle}_a^b$ Statistic}
This statistic represents the integrated average of the correlation function over an angular range. It provides a measure of the mean correlation, which is less sensitive to outliers than $S_a^b$.
\begin{DispWithArrows}[format=ll, displaystyle]
    \bar{\langle\xi\rangle}_a^b & =\frac{1}{b-a} \int_a^b C(\theta)d\cos(\theta) \\
    & = \frac{1}{b-a} \sum_l \frac{D_\ell}{2l(l+1)}\left[P_{l+1}(x)-P_{l-1}(x)\right]_a^b
\end{DispWithArrows}
This formulation is derived by substituting the series expansion of $C(\theta)$ and using the identity $\int P_\ell(x)dx = \frac{P_{\ell+1}(x) - P_{\ell-1}(x)}{2\ell+1}$.

\subsubsection{The $C(180)$ Statistic}
The correlation at $180^\circ$ is a special case that probes the largest observable scales in the universe. Physically, it compares the temperature of a point in the sky with the point directly opposite to it. This statistic is valuable because it is insensitive to the galactic mask used to remove foreground contamination, which primarily affects smaller angular scales. It is calculated by rotating and reflecting the CMB map to create an "opposite view" and then computing the correlation between the original and transformed maps.

\section{Methodology and Simulations}

\subsection{Cosmological Simulations with CAMB}
To test cosmological models, we generate theoretical power spectra using the \textit{Code for Anisotropies in the Microwave Background} (CAMB) \autocite{CAMB_GitHub}. CAMB takes a set of cosmological parameters and computes the expected CMB power spectrum for that "universe." The key parameters include:
\begin{itemize}[noitemsep,topsep=0pt]
  \item \textbf{ombh2 ($\Omega_b h^2$):} The physical baryon density parameter.
  \item \textbf{omch2 ($\Omega_c h^2$):} The physical cold dark matter density parameter.
  \item \textbf{cosmomc\_theta ($100\theta_{MC}$):} The angular size of the sound horizon at recombination.
  \item \textbf{tau ($\tau$):} The optical depth to reionization.
  \item \textbf{ns ($n_s$):} The scalar spectral index, which measures the scale dependence of the primordial power spectrum.
  \item \textbf{logA ($\ln(10^{10} A_s)$):} The amplitude of the primordial scalar perturbations.
  \item \textbf{omegak ($\Omega_k$):} The curvature density parameter, where $\Omega_k=0$ corresponds to a flat universe.
\end{itemize}

\subsection{Connecting Theory with Data}
The analysis pipeline connects the theoretical predictions with observational data from the Planck satellite \autocite{planck_legacy_archive, cosmoplanck_2020}. We use Markov Chain Monte Carlo (MCMC) chains from the Planck analysis, which provide distributions of probable cosmological parameters.

\begin{algorithm}[Simplified Analysis Pipeline]
The overall process can be summarized as follows:
\begin{enumerate}
    \item \textbf{Parameter Sampling:} Take a representative sample (e.g., the last 1000 points) from a converged MCMC chain of cosmological parameters provided by the Planck collaboration. Each point in the chain represents a complete set of parameters for a viable universe.
    \item \textbf{Theoretical Spectrum Generation:} For each set of parameters, use CAMB to compute the theoretical temperature power spectrum ($D_\ell$). This creates an ensemble of 1000 simulated power spectra.
    \item \textbf{Correlation Function Calculation:} For each simulated $D_\ell$, calculate the corresponding angular correlation function, $C(\theta)$.
    \item \textbf{Statistical Analysis:}
    \begin{itemize}
        \item Compute the mean and standard deviation of the ensemble of simulated power spectra and correlation functions. This provides a theoretical prediction with an associated uncertainty.
        \item Calculate the statistics ($S_a^b$, $\bar{\langle\xi\rangle}_a^b$, $C(180)$) for each of the 1000 simulated universes to generate a theoretical distribution for each statistic.
    \end{itemize}
    \item \textbf{Comparison with Observational Data:}
    \begin{itemize}
        \item Calculate the same statistics using the actual observed power spectrum from Planck.
        \item Compare the observed statistical values against the theoretical distributions generated in the previous step to quantify any tension or discrepancy, typically in terms of standard deviations ($\sigma$).
    \end{itemize}
\end{enumerate}
\end{algorithm}

\section{Results and Discussion}

The application of the statistical framework to the Planck data and simulated universes reveals several key insights. The following figures summarize the main results of this analysis.

\begin{figure}[H]
    \centering
    \includegraphics[width=0.7\textwidth]{pc.png}
    \caption{Dispersion of the simulated power spectra and angular correlation functions. The MCMC simulations show remarkable stability, with small variations in cosmological parameters leading to minimal deviation in the predicted functional forms.}
    \label{fig:esp-cor}
\end{figure}

\begin{figure}[H]
    \centering
    \includegraphics[width=0.7\textwidth]{s12.png}
    \caption{The $S_{1/2}$ statistic, as introduced by the WMAP collaboration \autocite{Spergel_2003}. The large error bars associated with this statistic prevent the establishment of a significant discrepancy with the standard cosmological model.}
    \label{fig:s12}
\end{figure}

\begin{figure}[H]
    \centering
    \includegraphics[width=0.7\textwidth]{xiv.png}
    \caption{The $\bar{\langle\xi\rangle}$ statistic defined in this work. This estimator, representing a mean value of the correlation function over specific angular ranges, provides a complementary view of the data.}
    \label{fig:xiv}
\end{figure}

\begin{figure}[H]
    \centering
    \includegraphics[width=0.7\textwidth]{hist.png}
    \caption{Histogram summarizing the significance of the discrepancies for the statistics defined in this work. While some angular ranges show hints of tension, the overall discrepancy is not statistically significant.}
    \label{fig:hist}
\end{figure}

The analysis indicates that while some tensions are observed, particularly with the base $F-\Lambda$CDM model, it is not possible to claim a significant new discrepancy. The study of the angular correlation function, while not revealing major tensions, provides a valuable cross-check and highlights the robustness of the standard model against this particular probe.

\appendix
\section{Error Propagation}
The error propagation assumes that the multipoles are independent of each other and have a Gaussian distribution. Also, it is assumed that the angle has no error at all.
\subsection{Error of the correlation function}
The error of the correlation function can be calculated according to this derivation 
\begin{DispWithArrows}[format=ll,displaystyle,subequations]
\Delta^2 C(\theta) 
&=\sum_l \Bigl(\frac{\partial C(\theta)}{\partial D_\ell}\,\Delta D_\ell\Bigr)^2 \\
&=\sum_l \Bigl( \frac{2\ell+1}{2\ell(\ell+1)} P_\ell(\cos(\theta))\Delta D_\ell\Bigr)^2\\
\Delta C(\theta) &=\sqrt{\sum_l \Bigl( \frac{2\ell+1}{2\ell(\ell+1)} P_\ell(\cos(\theta))\Delta D_\ell\Bigr)^2}
\end{DispWithArrows}

\subsection{Error of $S_a^b$}
\begin{DispWithArrows}[format=ll,displaystyle,subequations]
\Delta^2 S_a^b 
&=\sum_k \Bigl(\frac{\partial S_a^b}{\partial D_k}\,\Delta D_k\Bigr)^2 \\
&=\sum_k \Bigl(2 \sum_m \frac{2k+1}{2k(k+1)}D_m \frac{2m+1}{2m(m+1)}T_{km} \Delta D_k \Bigr)^2 \\
\Delta S_a^b &= 2 \sqrt{\sum_k \Bigl(\sum_m \frac{2k+1}{2k(k+1)}D_m \frac{2m+1}{2m(m+1)}T_{km} \Delta D_k \Bigr)^2}
\end{DispWithArrows}

\subsection{Error of $\bar{\langle\xi\rangle}_a^b$}
\begin{DispWithArrows}[format=ll,displaystyle,subequations]
\Delta^2 \bar{\langle\xi\rangle}_a^b 
&=\sum_l \Bigl(\frac{\partial \bar{\langle\xi\rangle}_a^b}{\partial D_\ell}\,\Delta D_\ell\Bigr)^2 \\
&=\sum_l \left( \frac{\Delta D_\ell}{2\ell(\ell+1)(b-a)}\left[P_{\ell+1}(x)-P_{\ell-1}(x)\right]_a^b \right)^2 \\
\Delta \bar{\langle\xi\rangle}_a^b &= \sqrt{\sum_l \left( \frac{\Delta D_\ell}{2\ell(\ell+1)(b-a)}\left[P_{\ell+1}(x)-P_{\ell-1}(x)\right]_a^b \right)^2}
\end{DispWithArrows}

\printbibliography

\end{document}
